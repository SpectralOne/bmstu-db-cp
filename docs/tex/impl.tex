\section{Технологическая часть}

В этом разделе проведён выбор средств реализации программного обеспечения, обусловлен выбор базы данных и СУБД, приведены листинги кода, реализующие необходимую для работы системы функциональность.

\subsection{Средства реализации программного обеспечения}

В качестве языка программирования был выбран функциональный язык \texttt{Clojure} \cite{clojure}, поддерживающий библиотеки JVM \cite{jvm}, подходящий для обработки больших объёмов данных и построения высоконагруженных систем, а также совместимый с ним функциональный язык \texttt{ClojureScript} \cite{cljs}, адаптированный для работы в браузере.

\subsection{Выбор базы данных и системы управления базой данных}

В данном подразделе рассмотрены популярные построчные СУБД, которые могут быть использованы для реализации хранения данных.

\subsubsection{Обзор СУБД с построчным хранением}

\noindent\textbf{PostgreSQL}

PostgreSQL \cite{postgresql} -- это свободно распространяемая объектно-реляционная система управления базами данных, наиболее развитая из открытых СУБД в мире и являющаяся реальной альтернативой коммерческим базам данных \cite{postgresql-fact}.

PostgreSQL предоставляет транзакции со свойствами атомарности, согласованности, изоляции, долговечности (\texttt{ACID} \cite{acid}), автоматически обновляемые представления, материализованные представления, триггеры, внешние ключи и хранимые процедуры. Данная СУБД предназначена для обработки ряда рабочих нагрузок, от отдельных компьютеров до хранилищ данных или веб-сервисов с множеством одновременных пользователей. 

Рассматриваемая СУБД управляет параллелизмом с помощью технологии параллельного доступа посредством многоверсионности (англ. \texttt{MVCC} \cite{mvcc}), которая предоставляет каждой транзакции <<снимок>> текущего состояния базы данных, позволяя вносить изменения, не затрагивая другие транзакции. Это в значительной степени устраняет необходимость в блокировках чтения (англ. \texttt{read lock} \cite{r-lock}) и гарантирует, что база данных поддерживает принципы \texttt{ACID}.\\

\noindent\textbf{Oracle Database}

Oracle Database \cite{oracle} -- объектно-реляционная система управления базами данных компании Oracle \cite{oracle-company}, являющаяся, на момент написания курсовой работы, самой популярной в мире \cite{oracle-popular}.

Oracle Database предоставляет транзакции, соответствующие принципам \texttt{ACID}, поддержку триггеров, хранимых процедур и внешние ключи. Особенностью рассматриваемой СУБД является функциональность работы с большими массивами данных, реализованная с использованием таких методов параллелизма, как механизм блокировки, обеспечивающий монопольное использование таблицы одной транзакцией и метод временных меток, обеспечивающий сериализацию и планирование транзакций.\\

\noindent\textbf{MySQL}

MySQL \cite{mysql} -- свободная реляционная система управления базами данных, разрабатываемая и поддерживаемая корпорацией Oracle.

Рассматриваемая СУБД имеет два основных движка хранения данных: \texttt{InnoDB} \cite{innodb} и \texttt{myISAM} \cite{myisam}. Движок \texttt{InnoDB} полностью совместим с принципами \texttt{ACID}, в отличии от движка \texttt{myISAM}. Параллелизм реализован с помощью механизма блокировок, который обеспечивает монопольный доступ к данным.

\subsubsection{Обзор NoSQL СУБД}

\noindent\textbf{Datomic}

Datomic \cite{datomic} -- это распределённая база данных класса NoSQL, реализующая язык логических операций \texttt{Datalog} \cite{datalog}.

Особенностью рассматриваемой СУБД является связь каждой транзакции с текущим временем, база данных хранит историю всех операций, что позволяет получить её состояние в любой момент времени, такой подход позволяет реализовать режим просмотра истории изменения данных, являющийся аналогом журналирования.

Datomic предоставляет транзакции, соответствующие принципам \texttt{ACID} и состоит из механизма изменения состояния базы данных -- транзактора, сервиса хранилища данных и пользовательского сервиса, который отсылает данные на транзактор. Оптимальная скорость работы достигается за счёт зарезервированных инструкций и консистентности данных, встроенного в транзактор механизма кэширования запросов как на чтение, так и на запись, и отсутствия планировщика транзакций. Унификация данных структурой вида <<ключ-значение>> позволяет достичь согласованности, в качестве ключа может быть как зарезервированное слово, так и атрибут сущности или хранимая процедура.

Главной особенностью является возможность пользователя полностью влиять на системные настройки и отсутствие явных правил формирования запроса. Неумелое обращение с данной СУБД может привести к результатам на порядок хуже, чем в традиционных реализациях, например, отсутствие планировщика транзакций может значительно замедлить выполнение составного запроса.\\

\noindent\textbf{Redis}

Redis \cite{redis} -- резидентная система управлениями базами данных класса NoSQL с открытым исходным кодом, используемая как для реализации кэшей и брокеров сообщений, так и для хранения данных. Основная структура представлена отношением вида <<ключ-значение>>.

Рассматриваемая СУБД хранит данные в оперативной памяти, снабжена механизмом <<снимков>> и журналирования, предоставляет операции для реализации механизма обмена сообщениями в шаблоне <<издатель-подписчик>>, позволяющие пользователю отслеживать изменения в системе, предоставляет встроенный механизм репликации данных, основанный на принципе master-slave \cite{master-slave}, поддерживает пакетную обработку команд и высокоуровневые операции над наборами данных.

Отличие Redis от других СУБД заключается в том, что значение, соответствующее определённому ключу, не ограничивается строковым типом. Поддерживаются следующие абстрактные типы данных:
\begin{itemize}
	\item строки;
	\item списки;
	\item множества;
	\item хеш-таблицы;
	\item упорядоченные множества.
\end{itemize}

\subsubsection*{Вывод}

В качестве СУБД была выбрана NoSQL Datomic с сервисом хранения PostgreSQL. Данное решение подходит для обработки большого объёма данных, удовлетворяет требованию в горизонтальном масштабировании, транзакции соответствуют свойствам \texttt{ACID}, отсутствует строгая типизация данных, а встроенный механизм кэширования позволяет обеспечить оптимальное время ожидания. Сравнительная характеристика СУБД представлена в таблице \ref{tab:dbs}.

\begin{table}[h]
	\begin{center}
		\caption{Сравнение СУБД}
		\label{tab:dbs}
		\begin{tabular}{|c|c|c|c|c|}
			\hline
			БД & Строгая типизация & ACID & Масштабируемость & Кэш \\
			\hline
			PostgreSQL & + & + & + & - \\ \hline
			Redis & - & - & $\pm$ & - \\ \hline
			MySQL & + & -  & + & - \\ \hline
			Oracle & + & + & + & - \\ \hline
			Datomic & - & + & + & + \\ \hline 
		\end{tabular}
	\end{center}
\end{table}

\subsection{Листинги кода}

В данном подразделе будут приведены листинги кода, реализующие отдельно взятую функциональность приложения и базы данных. Будут рассмотрены: функции создания схемы и пример её построения, функции доступа к данным, использующие обход графа по обратным ссылкам \cite{backref} с селектором, функция автоматического создания уникального кода и функция осуществления запроса к серверу приложения.

\subsubsection{Создание схемы базы данных}

Для упрощения создания схемы был реализован генератор схемы, позволяющий автоматически создавать сущность с предустановленными атрибутами и добавлять новые, передавая их в коротком формате, не используя пространство имён. Генератор схемы и пример создания сущности отражён в приложении \hyperref[app:schema]{А}. 

Учитывая особенности выбранной СУБД, ключевым словом, определяющим атрибут, является ключ \texttt{:db/ident}, а значением может быть любое ключевое слово, например, \texttt{:namespace/type}. Тип задаётся ключевым словом \\* \texttt{:db/valueType} и может принимать следующие значения:
\begin{itemize}
	\item \texttt{:db.type/string} -- строковый тип;
	\item \texttt{:db.type/boolean} -- логический тип;
	\item \texttt{:db.type/long} -- целочисленный тип;
	\item \texttt{:db.type/float} -- тип с плавающей точкой;
	\item \texttt{:db.type/keyword} -- тип ключевого слова;
	\item \texttt{:db.type/ref} -- ссылка на пользовательский тип или сущность.
\end{itemize}

\subsubsection{Доступ к атрибутам сущности базы данных}

Datomic предоставляет сущность в виде графа, где связь с другой сущностью реализована с помощью обратной ссылки. Для доступа к нужным атрибутам сущности был реализован механизм обхода графа по обратным ссылкам с использованием селектора атрибутов. Чтобы явно указать, какие поля сущности нужно получить, необходимо передать соответствующий селектор формата \texttt{\{ns [[attr op]]\}}, где \texttt{ns} -- пространство имён сущности, \texttt{attr} -- нужный атрибут, а \texttt{op} -- операция, определяемая типом атрибута: \texttt{atomic} по умолчанию, \texttt{pull-ref} -- обход графа по обратной ссылке. Реализация функции обхода графа сущности представлена в приложении \hyperref[app:db]{Б}.

\subsection{Хранимые процедуры}

В качестве идентификатора сущности используется уникальный код, который не может одновременно соответствовать более, чем одной сущности. Для упрощения создания уникального кода средствами языка \texttt{Clojure} была реализована хранимая процедура с использованием системной сущности счётчика, вызываемая при осуществлении транзакции добавления, чтобы установить необходимое уникальное значение в соответствующий атрибут сущности. Реализация автоматического создания кода приведена в приложении \hyperref[app:fns]{В}.

\subsection{Взаимодействие с приложением}

Взаимодействие с приложением осуществляется посредством метода вызова удалённых процедур (анг. Remote Procedure Call, \texttt{RPC} \cite{rpc}), состоящего в расширении механизма передачи управления и данных внутри программы, выполняющейся на одном узле, на передачу управления и данных через сеть. Средства удалённого вызова процедур предназначены для облегчения организации распределённых вычислений и создания распределенных клиент-серверных информационных систем. Наибольшая эффективность использования RPC достигается в тех приложениях, в которых существует интерактивная связь между удалёнными компонентами с небольшим временем ответов и относительно малым количеством передаваемых данных. Реализация удалённого вызова процедуры представлена в приложении \hyperref[app:rpc]{Г}.
