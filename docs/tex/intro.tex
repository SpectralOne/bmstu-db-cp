\specsection{Введение}

Рабочая программа дисциплины -- программа освоения учебного материала, соответствующая требованиям государственного образовательного стандарта высшего профессионального образования и учитывающая специфику \\* подготовки студентов по выбранному направлению или специальности. Разрабатывается для каждой дисциплины учебного плана всех реализуемых в университете основных образовательных программ \cite{rpd}.

Фонд оценочных средств -- совокупность оценочных материалов, а также описание форм и процедур, предназначенных для определения уровня достижения обучающимся установленных результатов обучения \cite{fos}.

Функции хранения, обработки и анализа информации, находящейся в рабочей программе дисциплины, могут быть реализованы в системах управления обучением (англ. \texttt{Learning Managment System} (LMS) \cite{lms}). Такой интерфейс может предоставить пользователю системы (в данном случае преподавателю) возможность получать и редактировать информацию о рабочей программе дисциплины.

Рабочая программа дисциплины представлена в виде документа формата \texttt{Microsoft Word} \cite{ms-word}, что накладывает ограничения на автоматизированную программную обработку и анализ информации, представленной в рабочей программе дисциплины.

Цель работы -- спроектировать базу данных для хранения компонентов нормативных документов и реализовать программное обеспечение для их создания, редактирования, удаления и составления на их основе рабочей программы дисциплины и фонда оценочных средств. 

Чтобы достигнуть поставленной цели, требуется решить следующие задачи:
\begin{itemize}
	\item провести анализ предметной области и готовых решений в области баз данных и систем управления базами данных;
	\item спроектировать базу данных, описать её компоненты и связи;
	\item реализовать базу данных и программное обеспечение для работы с ней;
	\item провести исследование устойчивости реализованной системы к высоким нагрузкам.
\end{itemize}